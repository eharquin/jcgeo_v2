\begin{frame}{Formal sum}
    \large
    \begin{center}
        \renewcommand{\arraystretch}{2.0} % more vertical space between rows
        \begin{tabular}{|c|c|}
            \hline
            \textbf{Linear Algebra} & \textbf{Geometric Algebra} \\ \hline
            $\vec{\mathbf{u}} =
                \begin{pmatrix}
                    a \\
                    b \\
                    c \\
                    \vdots
                \end{pmatrix}
                \textcolor{gray}{
                    \begin{pmatrix}
                        \textcolor{red}{e_x}   \\
                        \textcolor{green}{e_y} \\
                        \textcolor{blue}{e_z}  \\
                        \vdots
                    \end{pmatrix}
                }$
                                    &
            $\vec{\mathbf{u}} = a\, \textcolor{red}{e_x} + b\, \textcolor{green}{e_y} + c\, \textcolor{blue}{e_z} + \cdots$
            \\ \hline
        \end{tabular}
    \end{center}
\end{frame}


\begin{frame}[t]{Bivectors and Outer Product}

    \begin{center}
        \only<1-3>{\huge $\textcolor{u}{\vec{u}} \wedge \textcolor{v}{\vec{v}}$ \only<3-4>{$=\textcolor{A}{\vec{A}}$}}
        \only<4>{\huge $\textcolor{u}{\vec{u}} \wedge \textcolor{v}{\vec{v}}=-\textcolor{v}{\vec{v}} \wedge \textcolor{u}{\vec{u}}$}
        \only<5>{\huge $\textcolor{A}{\vec{A}}$}
        \only<6-7>{\huge $\|\textcolor{A}{\vec{A}}\|$ \only<7>{$=\sin(\textcolor{gray}{\theta}) \|\textcolor{u}{\vec{u}}\| \|\textcolor{v}{\vec{v}}\|$}}
        \only<8>{\huge $\textcolor{u}{\vec{u}} \wedge \textcolor{v}{\vec{v}} = 0$}
    \end{center}

    \begin{minipage}[c][0.6\textheight][c]{\linewidth}
        \centering
        \begin{tikzpicture}
            \coordinate (O) at (0,0);

            \def\xa{2.5}
            \def\ya{2.5}
            \coordinate (a) at (\xa,\ya);

            \def\xb{4}
            \def\yb{0}
            \coordinate (b) at (\xb,\yb);

            \visible<3>
            {
                \draw[ultra thick, A, ->, >=Triangle] ({\xa+\xb},{\ya}) -- (b);
                \draw[ultra thick, A, ->, >=Triangle] (a) -- ({\xa+\xb},{\ya});
                \draw[ultra thick, A, ->, >=Triangle] (b) -- (O);
                \draw[ultra thick, A, ->, >=Triangle] (O) -- (a);

                \draw[draw=none, fill=A, opacity=0.2] ({\xa+\xb},{\ya}) -- (b) -- (O) -- (a) -- cycle;
                % \node[A] at (3,1.25) {$\vec{A}$};
                \draw[A!60, thick,<-,>=Triangle] (3.75,1.25) arc[radius=0.5cm,start angle=0,delta angle=270];
                % point on the center of the parallelogram
            }

            \visible<1-3>
            {
                \draw[ultra thick, u, ->, >=Triangle] (O) -- (a);
            }

            \visible<1>
            {
                \draw[ultra thick, v, ->, >=Triangle] (O) -- (b);
            }

            \visible<2-3>
            {
                \draw[ultra thick, v, ->, >=Triangle] (a) -- ({\xa+\xb},{\ya});
            }


            % \visible<5>
            % {
            %     \draw[ultra thick, A, <-, >=Triangle] ({\xa+\xb},{\ya}) -- (b);
            %     \draw[ultra thick, A, <-, >=Triangle] (a) -- ({\xa+\xb},{\ya});
            %     \draw[ultra thick, A, <-, >=Triangle] (b) -- (O);
            %     \draw[ultra thick, A, <-, >=Triangle] (O) -- (a);

            %     \draw[draw=none, fill=A, opacity=0.2] ({\xa+\xb},{\ya}) -- (b) -- (O) -- (a) -- cycle;
            %     % \node[A] at (3,1.25) {$\vec{A}$};
            %     \draw[A!60, thick,->,>=Triangle] (3.75,1.25) arc[radius=0.5cm,start angle=0,delta angle=270];
            %     % point on the center of the parallelogram
            % }

            % \visible<4-5>
            % {
            %     \draw[ultra thick, v, ->, >=Triangle] (O) -- (b);
            %     \draw[ultra thick, u, ->, >=Triangle] (b) -- ({\xa+\xb},{\ya});
            % }

            \only<4>
            {
                \begin{scope}[shift={(-3,0)}]
                    \coordinate (O) at (0,0);

                    \def\xa{2.5}
                    \def\ya{2.5}
                    \coordinate (a) at (\xa,\ya);

                    \def\xb{4}
                    \def\yb{0}
                    \coordinate (b) at (\xb,\yb);

                    \draw[ultra thick, A, ->, >=Triangle] ({\xa+\xb},{\ya}) -- (b);
                    \draw[ultra thick, A, ->, >=Triangle] (a) -- ({\xa+\xb},{\ya});
                    \draw[ultra thick, A, ->, >=Triangle] (b) -- (O);
                    \draw[ultra thick, A, ->, >=Triangle] (O) -- (a);

                    \draw[draw=none, fill=A, opacity=0.2] ({\xa+\xb},{\ya}) -- (b) -- (O) -- (a) -- cycle;
                    % \node[A] at (3,1.25) {$\vec{A}$};
                    \draw[A!60, thick,<-,>=Triangle] (3.75,1.25) arc[radius=0.5cm,start angle=0,delta angle=270];
                    % point on the center of the parallelogram
                    \draw[ultra thick, u, ->, >=Triangle] (O) -- (a);
                    \draw[ultra thick, v, ->, >=Triangle] (a) -- ({\xa+\xb},{\ya});
                \end{scope}

                \begin{scope}[shift={(3,0)}]
                    \coordinate (O) at (0,0);

                    \def\xa{2.5}
                    \def\ya{2.5}
                    \coordinate (a) at (\xa,\ya);

                    \def\xb{4}
                    \def\yb{0}
                    \coordinate (b) at (\xb,\yb);

                    \draw[ultra thick, A, <-, >=Triangle] ({\xa+\xb},{\ya}) -- (b);
                    \draw[ultra thick, A, <-, >=Triangle] (a) -- ({\xa+\xb},{\ya});
                    \draw[ultra thick, A, <-, >=Triangle] (b) -- (O);
                    \draw[ultra thick, A, <-, >=Triangle] (O) -- (a);

                    \draw[draw=none, fill=A, opacity=0.2] ({\xa+\xb},{\ya}) -- (b) -- (O) -- (a) -- cycle;
                    % \node[A] at (3,1.25) {$\vec{A}$};
                    \draw[A!60, thick,->,>=Triangle] (3.75,1.25) arc[radius=0.5cm,start angle=0,delta angle=270];
                    % point on the center of the parallelogram

                    \draw[ultra thick, v, ->, >=Triangle] (O) -- (b);
                    \draw[ultra thick, u, ->, >=Triangle] (b) -- ({\xa+\xb},{\ya});
                \end{scope}

            }

            \visible<5>
            {
                \draw[ultra thick, A, ->, >=Triangle] ({\xa+\xb},{\ya}) -- (b);
                \draw[ultra thick, A, ->, >=Triangle] (a) -- ({\xa+\xb},{\ya});
                \draw[ultra thick, A, ->, >=Triangle] (b) -- (O);
                \draw[ultra thick, A, ->, >=Triangle] (O) -- (a);

                \draw[draw=none, fill=A, opacity=0.2] ({\xa+\xb},{\ya}) -- (b) -- (O) -- (a) -- cycle;
                % \node[A] at (3,1.25) {$\vec{A}$};
                \draw[A!60, thick,<-,>=Triangle] (3.75,1.25) arc[radius=0.5cm,start angle=0,delta angle=270];
                % point on the center of the parallelogram
            }

            \visible<6-7>
            {
                \draw[draw=none, fill=A, opacity=0.2] ({\xa+\xb},{\ya}) -- (b) -- (O) -- (a) -- cycle;
            }

            \visible<7>
            {
                \draw[ultra thick, u, ->, >=Triangle] (O) -- (a);
                \draw[ultra thick, v, ->, >=Triangle] (O) -- (b);

                \draw pic[
                        ultra thick,
                        gray,
                        draw=gray,
                        angle radius=1cm,
                        angle eccentricity=1.2,
                        "$\theta$"
                    ] {angle=b--O--a};
            }

            \visible<8>
            {
                \draw[ultra thick, v, ->, >=Triangle] (O) -- (b);
                \draw[ultra thick, u, ->, >=Triangle] (O) -- ({3.54},{0});
            }

            % \draw[red] (current bounding box.south west) rectangle (current bounding box.north east);

        \end{tikzpicture}

    \end{minipage}

    \centering
    \huge
    \only<4>{Anti-Commutativity!}
    \only<8>{Contract parallel dimensions!}


\end{frame}



\begin{frame}[t]{Outer Product properties}

    \begin{center}
        \only<1>{\Huge $\textcolor{u}{\vec{u}} \wedge \textcolor{v}{\vec{v}} =\textcolor{A}{\vec{A}}$}
    \end{center}

    \begin{columns}

        \column{0.5\textwidth}
        \only<1>
        {
            \begin{tabular}{rcl}
                $\lambda \wedge \textcolor{u}{\vec{u}}$                                                & $=$ & $\lambda\,\textcolor{u}{\vec{u}}$                                                                                 \\
                $(\textcolor{u}{\vec{u}} \wedge \textcolor{v}{\vec{v}}) \wedge \textcolor{w}{\vec{w}}$ & $=$ & $\textcolor{u}{\vec{u}} \wedge (\textcolor{v}{\vec{v}} \wedge \textcolor{w}{\vec{w}})$                            \\
                $\textcolor{u}{\vec{u}} \wedge (\textcolor{v}{\vec{v}} + \textcolor{w}{\vec{w}})$      & $=$ & $(\textcolor{u}{\vec{u}} \wedge \textcolor{v}{\vec{v}}) + (\textcolor{u}{\vec{u}} \wedge \textcolor{w}{\vec{w}})$ \\
                $\textcolor{u}{\vec{u}} \wedge (\lambda \textcolor{v}{\vec{v}})$                       & $=$ & $\lambda\,(\textcolor{u}{\vec{u}} \wedge \textcolor{v}{\vec{v}})$                                                 \\
                \\
                $\textcolor{u}{\vec{u}} \wedge \textcolor{v}{\vec{v}}$                                 & $=$ & $-\textcolor{v}{\vec{v}} \wedge \textcolor{u}{\vec{u}}$                                                           \\
                $\textcolor{u}{\vec{u}} \wedge \lambda \textcolor{u}{\vec{u}}$                         & $=$ & $0$                                                                                                               \\
            \end{tabular}
        }

        \vspace{2em}


        \column{0.5\textwidth}
        \begin{minipage}[c][0.6\textheight][c]{\linewidth} % [c] vertical alignment, [\textheight] height, [c] content alignment
            \centering
            \begin{tikzpicture}

                \visible<1>
                {
                    \begin{scope}[shift={(0,-2)}]
                        \coordinate (O) at (0,0);

                        \def\xa{2.5}
                        \def\ya{2.5}
                        \coordinate (a) at (\xa,\ya);

                        \def\xb{4}
                        \def\yb{0}
                        \coordinate (b) at (\xb,\yb);

                        \draw[ultra thick, A, ->, >=Triangle] ({\xa+\xb},{\ya}) -- (b);
                        \draw[ultra thick, A, ->, >=Triangle] (a) -- ({\xa+\xb},{\ya});
                        \draw[ultra thick, A, ->, >=Triangle] (b) -- (O);
                        \draw[ultra thick, A, ->, >=Triangle] (O) -- (a);

                        \draw[draw=none, fill=A, opacity=0.2] ({\xa+\xb},{\ya}) -- (b) -- (O) -- (a) -- cycle;
                        % \node[A] at (3,1.25) {$\vec{A}$};
                        \draw[A!60, thick,<-,>=Triangle] (3.75,1.25) arc[radius=0.5cm,start angle=0,delta angle=270];
                        % point on the center of the parallelogram

                        \draw[ultra thick, u, ->, >=Triangle] (O) -- (a);
                        \draw[ultra thick, v, ->, >=Triangle] (O) -- (b);
                    \end{scope}
                }
            \end{tikzpicture}
        \end{minipage}

    \end{columns}

\end{frame}


\begin{frame}{Bivector in 2d}
    \Large
    \[
        \textcolor{u}{\vec{u}} = a_1 \textcolor{red}{\mathbf{e}_x} + b_1 \textcolor{green}{\mathbf{e}_y}
        \qquad
        \textcolor{v}{\vec{v}} = a_2 \textcolor{red}{\mathbf{e}_x} + b_2 \textcolor{green}{\mathbf{e}_y}
    \]

    \begin{align*}
        \textcolor{u}{\vec{u}} \wedge \textcolor{v}{\vec{v}}
         & = (a_1 \textcolor{red}{\mathbf{e}_x} + b_1 \textcolor{green}{\mathbf{e}_y}) \wedge (a_2 \textcolor{red}{\mathbf{e}_x} + b_2 \textcolor{green}{\mathbf{e}_y}) \\
         & = a_1 a_2 (\textcolor{red}{\mathbf{e}_x} \wedge \textcolor{red}{\mathbf{e}_x})
        + a_1 b_2 (\textcolor{red}{\mathbf{e}_x} \wedge \textcolor{green}{\mathbf{e}_y})                                                                                \\
         & \quad + b_1 a_2 (\textcolor{green}{\mathbf{e}_y} \wedge \textcolor{red}{\mathbf{e}_x})
        + b_1 b_2 (\textcolor{green}{\mathbf{e}_y} \wedge \textcolor{green}{\mathbf{e}_y})                                                                              \\
         & = (a_1 b_2 - b_1 a_2)(\textcolor{red}{\mathbf{e}_x} \wedge \textcolor{green}{\mathbf{e}_y})                                                                  \\
         & = (a_1 b_2 - b_1 a_2) \textcolor{midredgreen}{\mathbf{e}_{xy}}
    \end{align*}
\end{frame}


\begin{frame}{Bivector in 3d}
    \Large
    \[
        \textcolor{u}{\vec{u}} = a_1 \textcolor{red}{\mathbf{e}_x} + b_1 \textcolor{green}{\mathbf{e}_y} + c_1 \textcolor{blue}{\mathbf{e}_z}
        \quad\quad
        \textcolor{v}{\vec{v}} = a_2 \textcolor{red}{\mathbf{e}_x} + b_2 \textcolor{green}{\mathbf{e}_y} + c_2 \textcolor{blue}{\mathbf{e}_z}
    \]

    \vspace{0.5em}

    \begin{align*}
        \textcolor{u}{\vec{u}} \wedge \textcolor{v}{\vec{v}} & =
        (a_1 b_2 - b_1 a_2)\, \textcolor{midredgreen}{\mathbf{e}_{xy}}
        + (c_1 a_2 - a_1 c_2)\, \textcolor{midredblue}{\mathbf{e}_{zx}}
        + (b_1 c_2 - c_1 b_2)\, \textcolor{midgreenblue}{\mathbf{e}_{yz}}
        \\
        \textcolor{u}{\vec{u}} \times \textcolor{v}{\vec{v}} & =
        (a_1 b_2 - b_1 a_2)\, \textcolor{blue}{\mathbf{e}_z}
        + (c_1 a_2 - a_1 c_2)\, \textcolor{green}{\mathbf{e}_y}
        + (b_1 c_2 - c_1 b_2)\, \textcolor{red}{\mathbf{e}_x}
    \end{align*}

    \vspace{0.5em}

    \textbf{Note}: looks like the cross product of \(\mathbb{R}^3\) but:
    \begin{itemize}
        \item defined in any dimension
        \item associativity: \((\textcolor{u}{\vec{u}} \wedge \textcolor{v}{\vec{v}}) \wedge \textcolor{w}{\vec{w}} = \textcolor{u}{\vec{u}} \wedge (\textcolor{v}{\vec{v}} \wedge \textcolor{w}{\vec{w}})\)
    \end{itemize}
\end{frame}




\begin{frame}[t]{Trivectors}

  \begin{center}
    \Huge
    $\textcolor{u}{\vec{u}} \wedge \textcolor{v}{\vec{v}} \wedge \textcolor{w}{\vec{w}} =\textcolor{AA}{A}$
  \end{center}

  \setlength{\columnsep}{20pt}
  \begin{columns}[T]
    \column{0.5\textwidth}

    \textbf{Properties}:
    \begin{itemize}
      \item Magnitude (Volume)
      \item Orientation
    \end{itemize}

    \textbf{Operations:}
    \begin{itemize}
      \item Addition
      \item Multiplication by a scalar
      \item \ldots
    \end{itemize}

    \column{0.5\textwidth}
    \begin{minipage}[c][0.5\textheight][c]{\linewidth} % [c] vertical alignment, [\textheight] height, [c] content alignment
      \centering
      \begin{tikzpicture}[remember picture]
        \useasboundingbox (-3,-3) rectangle (3,3);

        \begin{scope}[
            scale=2,
            view={60}{135},
            perspective={
                p={(-10,0,0)},
                q={(0,-10,0)},
                r={(0,0,-60)}
              },
            bullet/.style={circle,fill,inner sep=1pt},font=\sffamily,
            cube/.style={very thick,black},
            grid/.style={opacity=0.1, very thin,gray},
            vec/.style={->, >=triangle 45, ultra thick, yellow},
            axis/.style={dotted, thick},
            bivector/.style={draw=none,fill=yellow, opacity=0.2}]

          % Draw a grid in the x-y plane
          \foreach \x in {-0.5,0,...,1.5}
          \foreach \y in {-0.5,0,...,1.5}
            {
              \draw[opacity=0.1, very thin,gray] (tpp cs:\x,-0.5,0) -- (tpp cs:\x,1.5,0);
              \draw[opacity=0.1, very thin,gray] (tpp cs:-0.5,\y,0) -- (tpp cs:1.5,\y,0);
            }

          % Draw the axes
          \draw[axis] (tpp cs:-0.5,0,0) -- (tpp cs:1.5,0,0);
          \draw[axis] (tpp cs:0,-0.5,0) -- (tpp cs:0,1.5,0);
          \draw[axis] (tpp cs:0,0,0) -- (tpp cs:0,0,1.5);

          \coordinate (A) at (tpp cs:0,0,0);
          \coordinate (B) at (tpp cs:1,0,0);
          \coordinate (C) at (tpp cs:1,1,0);
          \coordinate (D) at (tpp cs:0,1,0);
          \coordinate (E) at (tpp cs:0,0,1);
          \coordinate (F) at (tpp cs:1,0,1);
          \coordinate (G) at (tpp cs:1,1,1);
          \coordinate (H) at (tpp cs:0,1,1);

          % Draw edges of the cube
          \draw[vec, opacity=0.5, u, >=latex] (A) -- (B);
          \draw[vec, opacity=0.5, v, >=latex] (B) -- (C);
          \draw[vec, opacity=0.5, AA, >=latex] (C) -- (D);
          \draw[vec, opacity=0.5, AA, >=latex] (D) -- (A);

          \draw[vec, opacity=0.5, AA, >=latex] (E) -- (F);
          \draw[vec, opacity=0.5, AA, >=latex] (F) -- (G);
          \draw[vec, opacity=0.5, AA, >=latex] (G) -- (H);
          \draw[vec, opacity=0.5, AA, >=latex] (H) -- (E);

          \draw[vec, opacity=0.5, AA, >=latex] (A) -- (E);
          \draw[vec, opacity=0.5, AA, >=latex] (B) -- (F);
          \draw[vec, opacity=0.5, w, >=latex] (C) -- (G);
          \draw[vec, opacity=0.5, AA, >=latex] (D) -- (H);

          % draw face of the cube
          \draw[bivector, AA] (B) -- (F) -- (G) -- (C) -- cycle;
          \draw[bivector, AA] (E) -- (F) -- (G) -- (H) -- cycle;
          \draw[bivector, AA] (C) -- (D) -- (H) -- (G) -- cycle;

        \end{scope}
        %\draw[red] (current bounding box.south west) rectangle (current bounding box.north east);
      \end{tikzpicture}
    \end{minipage}

  \end{columns}

\end{frame}




\begin{frame}{$k$-vectors and subspaces}
    \textbf{Set of vector spaces in 3D:}

    \vspace{1em}

    \[
        \left\{
        \underbrace{\mathbf{1}}_{\text{scalar}},\quad
        \underbrace{
            \textcolor{red}{\mathbf{e}_x},\
            \textcolor{green}{\mathbf{e}_y},\
            \textcolor{blue}{\mathbf{e}_z}
        }_{\text{vector space}},\quad
        \underbrace{
            \textcolor{midredgreen}{\mathbf{e}_{xy}},\
            \textcolor{midredblue}{\mathbf{e}_{zx}},\
            \textcolor{midgreenblue}{\mathbf{e}_{yz}}
        }_{\text{bivector space}},\quad
        \underbrace{
            \textcolor{mixAll}{\mathbf{e}_{xyz}}
        }_{\text{trivector space}}
        \right\}
    \]

    \vspace{1em}

    \textbf{Pseudo-scalar: }
    \(
    \textcolor{mixAll}{\mathbf{I}}_3 =
    \textcolor{red}{\mathbf{e}_x} \wedge
    \textcolor{green}{\mathbf{e}_y} \wedge
    \textcolor{blue}{\mathbf{e}_z} =
    \textcolor{mixAll}{\mathbf{e}_{xyz}}
    \)
\end{frame}




%%%%%%%%%%%%%%%%%%
\begin{frame}{Right complement and dual basis}
    ~\\~\\
    \textbf{Right complement:}~~~
    $\mathbf{e_{\{i\}} \wedge \mathbf{\overline{e}_{\{i\}}}} = ~ \textcolor{mixAll}{\mathbf{I}}_\mathrm{d}
        =\textcolor{red}{\mathbf{e}_{x}} \wedge
        \textcolor{green}{\mathbf{e}_{y}} \wedge
        \cdots$
    ~\\~\\
    \textcolor{gray}{\rule{\textwidth}{0.3pt}}
    ~\\
    \textbf{In 3D:}\\
    $$
        \textcolor{midredgreen}{\mathbf{e}_{xy}} \wedge \textcolor{blue}{\mathbf{\overline{e}}_{xy}} =
        \textcolor{red}{\mathbf{e}_x} \wedge
        \textcolor{green}{\mathbf{e}_y} \wedge
        \textcolor{blue}{\mathbf{e}_z} = \textcolor{mixAll}{\mathbf{I}}_3
    $$
    $$
        \underbrace{
            \textcolor{red}{\mathbf{e}_{x}} \wedge \textcolor{green}{\mathbf{e}_{y}}
        }_{\textcolor{midredgreen}{\mathbf{e}_{xy}}}
        \wedge
        \Big(
        \underbrace{
            \textcolor{blue}{\mathbf{e}_{z}}
        }_{ \textcolor{blue}{\mathbf{\overline{e}}_{xy}} }
        \Big)
        = \textcolor{mixAll}{\mathbf{I}}_3
    $$
    ~\\
\end{frame}



%%%%%%%%%%%%%%%%%%
\begin{frame}{Right complement and dual basis}
    ~\\~\\
    \textbf{Right complement:}~~~
    $\mathbf{e_{\{i\}} \wedge \mathbf{\overline{e}_{\{i\}}}} = ~ \textcolor{mixAll}{\mathbf{I}}_\mathrm{d}
        =\textcolor{red}{\mathbf{e}_{x}} \wedge
        \textcolor{green}{\mathbf{e}_{y}} \wedge
        \cdots$

    ~\\~\\
    \textcolor{gray}{\rule{\textwidth}{0.3pt}}
    ~\\
    \textbf{In 3D:}\\
    \begin{center}
        \begin{tabular}{c}
            $\mathbf{1} = \textcolor{mixAll}{\mathbf{\overline{e}}_{xyz}}$
        \end{tabular}
        \begin{tabular}{r@{=}r}
            \textcolor{red}{$\mathbf{e}_{x}$}   & $-\textcolor{midgreenblue}{\mathbf{\overline{e}}_{yz}}$ \\
            \textcolor{green}{$\mathbf{e}_{y}$} & $\textcolor{midredblue}{\mathbf{\overline{e}}_{xz}}$    \\
            \textcolor{blue}{$\mathbf{e}_{z}$}  & $-\textcolor{midredgreen}{\mathbf{\overline{e}}_{xy}}$  \\
        \end{tabular}
        \begin{tabular}{r@{=}r}
            \textcolor{midredgreen}{$\mathbf{e}_{xy}$}  & $\textcolor{blue}{\mathbf{\overline{e}}_{z}}$   \\
            \textcolor{midredblue}{$\mathbf{e}_{xz}$}   & $-\textcolor{green}{\mathbf{\overline{e}}_{y}}$ \\
            \textcolor{midgreenblue}{$\mathbf{e}_{yz}$} & $\textcolor{red}{\mathbf{\overline{e}}_{x}}$    \\
        \end{tabular}
        \begin{tabular}{c}
            $\textcolor{mixAll}{\mathbf{e}_{xyz}} = \mathbf{\overline{1}}$
        \end{tabular}
    \end{center}
\end{frame}



%%%%%%%%%%%%%%%%%%
\begin{frame}{Anti-wedge Product}
    \begin{itemize}
        \item regressive / anti-wedge product
        \item written $\mathbf{{u}} \vee \mathbf{{v}}$ (read $\mathbf{{u}}$ ``anti-wedge'' $\mathbf{{v}}$)
        \item property: $\mathbf{{u}} \vee \mathbf{{v}} = \overline{\overline{\mathbf{u}} \wedge \overline{\mathbf{v}}}$
    \end{itemize}
    ~\\~\\
    \textbf{Example in 3D:}
    $$
        \textcolor{midgreenblue}{\mathbf{e}_{yz}} \vee \textcolor{midredblue}{\mathbf{e}_{xz}}
        = \overline{
            \textcolor{red}{\mathbf{\overline{e}}_{yz}} \wedge
            \textcolor{green}{\mathbf{\overline{e}}_{xz}}
        }
        = -\overline{
            \textcolor{red}{\mathbf{e}_{x}} \wedge
            \textcolor{green}{\mathbf{e}_{y}}
        }
        = -\textcolor{midredgreen}{\mathbf{\overline{e}}_{xy}}
        = -\textcolor{blue}{\mathbf{e}_{z}}
    $$
\end{frame}
