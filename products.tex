\begin{frame}[t]{Bivectors and Outer Product}

    \begin{center}
        \only<1-3>{\huge $\textcolor{u}{\vec{u}} \wedge \textcolor{v}{\vec{v}}$ \only<3-4>{$=\textcolor{A}{\vec{A}}$}}
        \only<4>{\huge $\textcolor{u}{\vec{u}} \wedge \textcolor{v}{\vec{v}}=-\textcolor{v}{\vec{v}} \wedge \textcolor{u}{\vec{u}}$}
        \only<5>{\huge $\textcolor{A}{\vec{A}}$}
        \only<6-7>{\huge $\|\textcolor{A}{\vec{A}}\|$ \only<7>{$=\sin(\textcolor{gray}{\theta}) \|\textcolor{u}{\vec{u}}\| \|\textcolor{v}{\vec{v}}\|$}}
        \only<8>{\huge $\textcolor{u}{\vec{u}} \wedge \textcolor{v}{\vec{v}} = 0$}
    \end{center}

    \begin{minipage}[c][0.6\textheight][c]{\linewidth}
        \centering
        \begin{tikzpicture}
            \coordinate (O) at (0,0);

            \def\xa{2.5}
            \def\ya{2.5}
            \coordinate (a) at (\xa,\ya);

            \def\xb{4}
            \def\yb{0}
            \coordinate (b) at (\xb,\yb);

            \visible<3>
            {
                \draw[ultra thick, A, ->, >=Triangle] ({\xa+\xb},{\ya}) -- (b);
                \draw[ultra thick, A, ->, >=Triangle] (a) -- ({\xa+\xb},{\ya});
                \draw[ultra thick, A, ->, >=Triangle] (b) -- (O);
                \draw[ultra thick, A, ->, >=Triangle] (O) -- (a);

                \draw[draw=none, fill=A, opacity=0.2] ({\xa+\xb},{\ya}) -- (b) -- (O) -- (a) -- cycle;
                % \node[A] at (3,1.25) {$\vec{A}$};
                \draw[A!60, thick,<-,>=Triangle] (3.75,1.25) arc[radius=0.5cm,start angle=0,delta angle=270];
                % point on the center of the parallelogram
            }

            \visible<1-3>
            {
                \draw[ultra thick, u, ->, >=Triangle] (O) -- (a);
            }

            \visible<1>
            {
                \draw[ultra thick, v, ->, >=Triangle] (O) -- (b);
            }

            \visible<2-3>
            {
                \draw[ultra thick, v, ->, >=Triangle] (a) -- ({\xa+\xb},{\ya});
            }


            % \visible<5>
            % {
            %     \draw[ultra thick, A, <-, >=Triangle] ({\xa+\xb},{\ya}) -- (b);
            %     \draw[ultra thick, A, <-, >=Triangle] (a) -- ({\xa+\xb},{\ya});
            %     \draw[ultra thick, A, <-, >=Triangle] (b) -- (O);
            %     \draw[ultra thick, A, <-, >=Triangle] (O) -- (a);

            %     \draw[draw=none, fill=A, opacity=0.2] ({\xa+\xb},{\ya}) -- (b) -- (O) -- (a) -- cycle;
            %     % \node[A] at (3,1.25) {$\vec{A}$};
            %     \draw[A!60, thick,->,>=Triangle] (3.75,1.25) arc[radius=0.5cm,start angle=0,delta angle=270];
            %     % point on the center of the parallelogram
            % }

            % \visible<4-5>
            % {
            %     \draw[ultra thick, v, ->, >=Triangle] (O) -- (b);
            %     \draw[ultra thick, u, ->, >=Triangle] (b) -- ({\xa+\xb},{\ya});
            % }

            \only<4>
            {
                \begin{scope}[shift={(-3,0)}]
                    \coordinate (O) at (0,0);

                    \def\xa{2.5}
                    \def\ya{2.5}
                    \coordinate (a) at (\xa,\ya);

                    \def\xb{4}
                    \def\yb{0}
                    \coordinate (b) at (\xb,\yb);

                    \draw[ultra thick, A, ->, >=Triangle] ({\xa+\xb},{\ya}) -- (b);
                    \draw[ultra thick, A, ->, >=Triangle] (a) -- ({\xa+\xb},{\ya});
                    \draw[ultra thick, A, ->, >=Triangle] (b) -- (O);
                    \draw[ultra thick, A, ->, >=Triangle] (O) -- (a);

                    \draw[draw=none, fill=A, opacity=0.2] ({\xa+\xb},{\ya}) -- (b) -- (O) -- (a) -- cycle;
                    % \node[A] at (3,1.25) {$\vec{A}$};
                    \draw[A!60, thick,<-,>=Triangle] (3.75,1.25) arc[radius=0.5cm,start angle=0,delta angle=270];
                    % point on the center of the parallelogram
                    \draw[ultra thick, u, ->, >=Triangle] (O) -- (a);
                    \draw[ultra thick, v, ->, >=Triangle] (a) -- ({\xa+\xb},{\ya});
                \end{scope}

                \begin{scope}[shift={(3,0)}]
                    \coordinate (O) at (0,0);

                    \def\xa{2.5}
                    \def\ya{2.5}
                    \coordinate (a) at (\xa,\ya);

                    \def\xb{4}
                    \def\yb{0}
                    \coordinate (b) at (\xb,\yb);

                    \draw[ultra thick, A, <-, >=Triangle] ({\xa+\xb},{\ya}) -- (b);
                    \draw[ultra thick, A, <-, >=Triangle] (a) -- ({\xa+\xb},{\ya});
                    \draw[ultra thick, A, <-, >=Triangle] (b) -- (O);
                    \draw[ultra thick, A, <-, >=Triangle] (O) -- (a);

                    \draw[draw=none, fill=A, opacity=0.2] ({\xa+\xb},{\ya}) -- (b) -- (O) -- (a) -- cycle;
                    % \node[A] at (3,1.25) {$\vec{A}$};
                    \draw[A!60, thick,->,>=Triangle] (3.75,1.25) arc[radius=0.5cm,start angle=0,delta angle=270];
                    % point on the center of the parallelogram

                    \draw[ultra thick, v, ->, >=Triangle] (O) -- (b);
                    \draw[ultra thick, u, ->, >=Triangle] (b) -- ({\xa+\xb},{\ya});
                \end{scope}

            }

            \visible<5>
            {
                \draw[ultra thick, A, ->, >=Triangle] ({\xa+\xb},{\ya}) -- (b);
                \draw[ultra thick, A, ->, >=Triangle] (a) -- ({\xa+\xb},{\ya});
                \draw[ultra thick, A, ->, >=Triangle] (b) -- (O);
                \draw[ultra thick, A, ->, >=Triangle] (O) -- (a);

                \draw[draw=none, fill=A, opacity=0.2] ({\xa+\xb},{\ya}) -- (b) -- (O) -- (a) -- cycle;
                % \node[A] at (3,1.25) {$\vec{A}$};
                \draw[A!60, thick,<-,>=Triangle] (3.75,1.25) arc[radius=0.5cm,start angle=0,delta angle=270];
                % point on the center of the parallelogram
            }

            \visible<6-7>
            {
                \draw[draw=none, fill=A, opacity=0.2] ({\xa+\xb},{\ya}) -- (b) -- (O) -- (a) -- cycle;
            }

            \visible<7>
            {
                \draw[ultra thick, u, ->, >=Triangle] (O) -- (a);
                \draw[ultra thick, v, ->, >=Triangle] (O) -- (b);

                \draw pic[
                        ultra thick,
                        gray,
                        draw=gray,
                        angle radius=1cm,
                        angle eccentricity=1.2,
                        "$\theta$"
                    ] {angle=b--O--a};
            }

            \visible<8>
            {
                \draw[ultra thick, v, ->, >=Triangle] (O) -- (b);
                \draw[ultra thick, u, ->, >=Triangle] (O) -- ({3.54},{0});
            }

            % \draw[red] (current bounding box.south west) rectangle (current bounding box.north east);

        \end{tikzpicture}

    \end{minipage}

    \centering
    \huge
    \only<4>{Anti-Commutativity!}
    \only<8>{Contract parallel dimensions!}


\end{frame}



\begin{frame}[t]{Bivectors and Outer Product}

    \begin{center}
        \only<1-2>{\Huge $\textcolor{u}{\vec{u}} \wedge \textcolor{v}{\vec{v}} =\textcolor{A}{\vec{A}}$}
    \end{center}

    \begin{columns}

        \column{0.5\textwidth}
        \only<1-2>
        {
            \begin{tabular}{rcl}
                $\lambda \wedge \textcolor{u}{\vec{u}}$                                                & $=$ & $\lambda\,\textcolor{u}{\vec{u}}$                                                                                 \\
                $(\textcolor{u}{\vec{u}} \wedge \textcolor{v}{\vec{v}}) \wedge \textcolor{w}{\vec{w}}$ & $=$ & $\textcolor{u}{\vec{u}} \wedge (\textcolor{v}{\vec{v}} \wedge \textcolor{w}{\vec{w}})$                            \\
                $\textcolor{u}{\vec{u}} \wedge (\textcolor{v}{\vec{v}} + \textcolor{w}{\vec{w}})$      & $=$ & $(\textcolor{u}{\vec{u}} \wedge \textcolor{v}{\vec{v}}) + (\textcolor{u}{\vec{u}} \wedge \textcolor{w}{\vec{w}})$ \\
                $\textcolor{u}{\vec{u}} \wedge (\lambda \textcolor{v}{\vec{v}})$                       & $=$ & $\lambda\,(\textcolor{u}{\vec{u}} \wedge \textcolor{v}{\vec{v}})$                                                 \\
                \\
                $\textcolor{u}{\vec{u}} \wedge \textcolor{v}{\vec{v}}$                                 & $=$ & $-\textcolor{v}{\vec{v}} \wedge \textcolor{u}{\vec{u}}$                                                           \\
                $\textcolor{u}{\vec{u}} \wedge \lambda \textcolor{u}{\vec{u}}$                         & $=$ & $0$                                                                                                               \\
            \end{tabular}
        }

        \vspace{2em}

        \only<2>
        {
            \begin{alertblock}{Limitations (again)}
                Not enough information about the vectors
            \end{alertblock}
        }


        \column{0.5\textwidth}
        \begin{minipage}[c][0.6\textheight][c]{\linewidth} % [c] vertical alignment, [\textheight] height, [c] content alignment
            \centering
            \begin{tikzpicture}

                \visible<1>
                {
                    \begin{scope}[shift={(0,-2)}]
                        \coordinate (O) at (0,0);

                        \def\xa{2.5}
                        \def\ya{2.5}
                        \coordinate (a) at (\xa,\ya);

                        \def\xb{4}
                        \def\yb{0}
                        \coordinate (b) at (\xb,\yb);

                        \draw[ultra thick, A, ->, >=Triangle] ({\xa+\xb},{\ya}) -- (b);
                        \draw[ultra thick, A, ->, >=Triangle] (a) -- ({\xa+\xb},{\ya});
                        \draw[ultra thick, A, ->, >=Triangle] (b) -- (O);
                        \draw[ultra thick, A, ->, >=Triangle] (O) -- (a);

                        \draw[draw=none, fill=A, opacity=0.2] ({\xa+\xb},{\ya}) -- (b) -- (O) -- (a) -- cycle;
                        % \node[A] at (3,1.25) {$\vec{A}$};
                        \draw[A!60, thick,<-,>=Triangle] (3.75,1.25) arc[radius=0.5cm,start angle=0,delta angle=270];
                        % point on the center of the parallelogram

                        \draw[ultra thick, u, ->, >=Triangle] (O) -- (a);
                        \draw[ultra thick, v, ->, >=Triangle] (O) -- (b);
                    \end{scope}
                }


                \visible<2>
                {
                    \begin{scope}[scale=0.6, shift={(3,0)}]
                        \coordinate (O) at (0,0);

                        \def\xa{2.5}
                        \def\ya{2.5}
                        \coordinate (a) at (\xa,\ya);

                        \def\xb{4}
                        \def\yb{0}
                        \coordinate (b) at (\xb,\yb);

                        \draw[ultra thick, A, ->, >=Triangle] ({\xa+\xb},{\ya}) -- (b);
                        \draw[ultra thick, A, ->, >=Triangle] (a) -- ({\xa+\xb},{\ya});
                        \draw[ultra thick, A, ->, >=Triangle] (b) -- (O);
                        \draw[ultra thick, A, ->, >=Triangle] (O) -- (a);

                        \draw[draw=none, fill=A, opacity=0.2] ({\xa+\xb},{\ya}) -- (b) -- (O) -- (a) -- cycle;
                        % \node[A] at (3,1.25) {$\vec{A}$};
                        \draw[A!60, thick,<-,>=Triangle] (3.75,1.25) arc[radius=0.5cm,start angle=0,delta angle=270];
                        % point on the center of the parallelogram

                        \draw[ultra thick, u, ->, >=Triangle] (O) -- (a);
                        \draw[ultra thick, v, ->, >=Triangle] (O) -- (b);

                        \node at (3.25,-1.4) {\huge$=$};

                        \begin{scope}[shift={(1.25,-5)}]
                            \coordinate (O) at (0,0);

                            \def\xa{0}
                            \def\ya{2.5}
                            \coordinate (a) at (\xa,\ya);

                            \def\xb{4}
                            \def\yb{0}
                            \coordinate (b) at (\xb,\yb);

                            \draw[ultra thick, A, ->, >=Triangle] ({\xa+\xb},{\ya}) -- (b);
                            \draw[ultra thick, A, ->, >=Triangle] (a) -- ({\xa+\xb},{\ya});
                            \draw[ultra thick, A, ->, >=Triangle] (b) -- (O);
                            \draw[ultra thick, A, ->, >=Triangle] (O) -- (a);

                            \draw[draw=none, fill=A, opacity=0.2] ({\xa+\xb},{\ya}) -- (b) -- (O) -- (a) -- cycle;
                            % \node[A] at (3,1.25) {$\vec{A}$};
                            \draw[A!60, thick,<-,>=Triangle] (2.5,1.25) arc[radius=0.5cm,start angle=0,delta angle=270];
                            % point on the center of the parallelogram

                            \draw[ultra thick, u, ->, >=Triangle] (O) -- (a);
                            \draw[ultra thick, v, ->, >=Triangle] (O) -- (b);

                        \end{scope}
                    \end{scope}
                }
            \end{tikzpicture}
        \end{minipage}

    \end{columns}

\end{frame}


\begin{frame}{Bivector in 2D}
    \Large
    \[
        \textcolor{u}{\vec{u}} = a_1 \textcolor{red}{\mathbf{e}_x} + b_1 \textcolor{green}{\mathbf{e}_y}
        \qquad
        \textcolor{v}{\vec{v}} = a_2 \textcolor{red}{\mathbf{e}_x} + b_2 \textcolor{green}{\mathbf{e}_y}
    \]

    \begin{align*}
        \textcolor{u}{\vec{u}} \wedge \textcolor{v}{\vec{v}}
         & = (a_1 \textcolor{red}{\mathbf{e}_x} + b_1 \textcolor{green}{\mathbf{e}_y}) \wedge (a_2 \textcolor{red}{\mathbf{e}_x} + b_2 \textcolor{green}{\mathbf{e}_y}) \\
         & = a_1 a_2 (\textcolor{red}{\mathbf{e}_x} \wedge \textcolor{red}{\mathbf{e}_x})
        + a_1 b_2 (\textcolor{red}{\mathbf{e}_x} \wedge \textcolor{green}{\mathbf{e}_y})                                                                                \\
         & \quad + b_1 a_2 (\textcolor{green}{\mathbf{e}_y} \wedge \textcolor{red}{\mathbf{e}_x})
        + b_1 b_2 (\textcolor{green}{\mathbf{e}_y} \wedge \textcolor{green}{\mathbf{e}_y})                                                                              \\
         & = (a_1 b_2 - b_1 a_2)(\textcolor{red}{\mathbf{e}_x} \wedge \textcolor{green}{\mathbf{e}_y})                                                                  \\
         & = (a_1 b_2 - b_1 a_2) \textcolor{midredgreen}{\mathbf{e}_{xy}}
    \end{align*}
\end{frame}


\begin{frame}{Bivector in 3D}
    \Large
    \[
        \textcolor{u}{\vec{u}} = a_1 \textcolor{red}{\mathbf{e}_x} + b_1 \textcolor{green}{\mathbf{e}_y} + c_1 \textcolor{blue}{\mathbf{e}_z}
        \quad\quad
        \textcolor{v}{\vec{v}} = a_2 \textcolor{red}{\mathbf{e}_x} + b_2 \textcolor{green}{\mathbf{e}_y} + c_2 \textcolor{blue}{\mathbf{e}_z}
    \]

    \vspace{0.5em}

    \begin{align*}
        \textcolor{u}{\vec{u}} \wedge \textcolor{v}{\vec{v}} & =
        (a_1 b_2 - b_1 a_2)\, \textcolor{midredgreen}{\mathbf{e}_{xy}}
        + (c_1 a_2 - a_1 c_2)\, \textcolor{midredblue}{\mathbf{e}_{zx}}
        + (b_1 c_2 - c_1 b_2)\, \textcolor{midgreenblue}{\mathbf{e}_{yz}}
        \\
        \textcolor{u}{\vec{u}} \times \textcolor{v}{\vec{v}} & =
        (a_1 b_2 - b_1 a_2)\, \textcolor{blue}{\mathbf{e}_z}
        + (c_1 a_2 - a_1 c_2)\, \textcolor{green}{\mathbf{e}_y}
        + (b_1 c_2 - c_1 b_2)\, \textcolor{red}{\mathbf{e}_x}
    \end{align*}

    \vspace{0.5em}

    \textbf{Note}: looks like the cross product of \(\mathbb{R}^3\) but:
    \begin{itemize}
        \item is actually defined in any dimension
        \item is associative: \((\textcolor{u}{\vec{u}} \wedge \textcolor{v}{\vec{v}}) \wedge \textcolor{w}{\vec{w}} = \textcolor{u}{\vec{u}} \wedge (\textcolor{v}{\vec{v}} \wedge \textcolor{w}{\vec{w}})\)
    \end{itemize}
\end{frame}

