\begin{frame}{Metric $\mathbb{R}_{3,0,1}$}
    \textbf{3D Projective geometric algebra:} \quad $\mathbb{R}_{3,0,1}$

    \begin{columns}
        \column{0.5\textwidth}
        \begin{center}
            \begin{tabular}{c|cccc}
                                                  & $\textcolor{red}{\mathbf{e}_x}$ & $\textcolor{green}{\mathbf{e}_y}$ & $\textcolor{blue}{\mathbf{e}_z}$ & $\textcolor{gray}{\mathbf{e}_0}$ \\ \hline
                $\textcolor{red}{\mathbf{e}_x}$   & 1                               & \textcolor{gray}{0}               & \textcolor{gray}{0}              & \textcolor{gray}{0}              \\
                $\textcolor{green}{\mathbf{e}_y}$ & \textcolor{gray}{0}             & 1                                 & \textcolor{gray}{0}              & \textcolor{gray}{0}              \\
                $\textcolor{blue}{\mathbf{e}_z}$  & \textcolor{gray}{0}             & \textcolor{gray}{0}               & 1                                & \textcolor{gray}{0}              \\
                $\textcolor{gray}{\mathbf{e}_0}$  & \textcolor{gray}{0}             & \textcolor{gray}{0}               & \textcolor{gray}{0}              & 0                                \\
            \end{tabular}
        \end{center}
        \column{0.5\textwidth}

        \begin{align*}
            \mathbf{p}       & = x\,\textcolor{red}{\mathbf{e}_x} + y\,\textcolor{green}{\mathbf{e}_y} + z\,\textcolor{blue}{\mathbf{e}_z} + w\,\textcolor{gray}{\mathbf{e}_0} \\
            \vec{\mathbf{u}} & = a\,\textcolor{red}{\mathbf{e}_x} + b\,\textcolor{green}{\mathbf{e}_y} + c\,\textcolor{blue}{\mathbf{e}_z}
        \end{align*}

    \end{columns}

\end{frame}




\begin{frame}[t]{Line from points and vectors}

    $$
        \mathbf{p} =
        x\textcolor{red}{\mathbf{e}_x} +
        y\textcolor{green}{\mathbf{e}_y} +
        z\textcolor{blue}{\mathbf{e}_z} +
        w\textcolor{gray}{\mathbf{e}_0}
        \qquad
        \vec{\mathbf{u}} =
        a\textcolor{red}{\mathbf{e}_x} +
        b\textcolor{green}{\mathbf{e}_y} +
        c\textcolor{blue}{\mathbf{e}_z}
    $$


    \vspace{0.5em}

    % Two-column layout
    \begin{columns}
        \column{0.5\textwidth}
        \centering
        \visible<2->{$\mathbf{L} = $} $\mathbf{p}_1 \wedge \only<1-2>{\mathbf{p}_2} \only<3->{\textcolor{u}{\vec{\mathbf{u}}}}$


        \column{0.5\textwidth}
        \begin{center}
            \begin{tikzpicture}
                \useasboundingbox (-2,-1) rectangle (2,1);

                \filldraw (-1.5, -0.643) circle (2pt) node[above left, scale=0.8] {$\mathbf{p}_1$};

                \visible<1-2>{\filldraw (1.5, 0.643) circle (2pt) node[above left, scale=0.8] {$\mathbf{p}_2$};}
                \visible<3->{\draw[u,>=Triangle,->, ultra thick] (-1.5, -0.643) -- (1.5, 0.643) node[midway, below] {$\vec{\mathbf{u}}$};}
                \only<2->{
                    \begin{scope}[on background layer]
                        \def\xa{3.5}
                        \def\ya{1.5}
                        \def\scaleFactor{5}
                        \draw[ultra thick, gray, overlay] ({-\scaleFactor*\xa}, {- \scaleFactor*\ya}) -- ({\scaleFactor*\xa}, {\scaleFactor*\ya});
                    \end{scope}
                }


                %\draw[red] (current bounding box.south west) rectangle (current bounding box.north east);
            \end{tikzpicture}
        \end{center}
    \end{columns}

    \vspace{3em}

    \begin{center}
        \visible<4->
        {
            $$
                \mathbf{p} \in \mathbf{L}
                \qquad \Leftrightarrow \qquad
                \mathbf{L} \wedge \mathbf{p} = 0
            $$
        }
        \visible<5->{
            \ldots corresponds to Plücker lines!
        }
    \end{center}
\end{frame}



\begin{frame}[t]{Plane from points and vectors}

    $$
        \mathbf{p} =
        x\textcolor{red}{\mathbf{e}_x} +
        y\textcolor{green}{\mathbf{e}_y} +
        z\textcolor{blue}{\mathbf{e}_z} +
        w\textcolor{gray}{\mathbf{e}_0}
        \qquad
        \vec{\mathbf{u}} =
        a\textcolor{red}{\mathbf{e}_x} +
        b\textcolor{green}{\mathbf{e}_y} +
        c\textcolor{blue}{\mathbf{e}_z}
    $$


    \vspace{0.5em}

    % Two-column layout
    \begin{columns}[T]
        \column{0.5\textwidth}

        \begin{align*}
            \mathbf{P} & = \mathbf{p}_1 \wedge \mathbf{p}_2 \wedge \mathbf{p}_3                                       \\
            \mathbf{P} & = \mathbf{p}_1 \wedge \mathbf{p}_2 \wedge \textcolor{u}{\vec{\mathbf{u}}}                    \\
            \mathbf{P} & = \mathbf{p}_1 \wedge \textcolor{u}{\vec{\mathbf{u}}} \wedge \textcolor{v}{\vec{\mathbf{v}}}
        \end{align*}

        \visible<5->{
            \centering{\textcolor{gray}{\rule{\linewidth}{0.3pt}}}
            $$\mathbf{p}\in \mathbf{P} ~~~\Leftrightarrow~~~ \mathbf{P} \wedge \mathbf{p} = 0 $$
        }%



        \column{0.5\textwidth}
        \begin{center}
            \begin{tikzpicture}
                \useasboundingbox (-3,-3) rectangle (3,2);

                \begin{scope}[
                        scale=1.5,
                        view={55}{135},
                        perspective={
                                p={(-7,0,0)},
                                q={(0,-7,0)},
                                r={(0,0,60)}
                            },
                        bullet/.style={circle,fill,inner sep=1pt},font=\sffamily,
                        cube/.style={very thick,black},
                        grid/.style={opacity=0.1, very thin,gray},
                        vec/.style={->, >=triangle 45, ultra thick, yellow},
                        axis/.style={dotted, thick},
                        bivector/.style={draw=none,fill=yellow, opacity=0.2}]

                    % Draw a grid in the x-y plane
                    \foreach \x in {-0.5,0,...,1.5}
                    \foreach \y in {-0.5,0,...,1.5}
                        {
                            \draw[opacity=0.1, very thin,gray] (tpp cs:\x,-0.5,0) -- (tpp cs:\x,1.5,0);
                            \draw[opacity=0.1, very thin,gray] (tpp cs:-0.5,\y,0) -- (tpp cs:1.5,\y,0);
                        }

                    % Draw the axes
                    \draw[axis] (tpp cs:-0.5,0,0) -- (tpp cs:1.5,0,0);
                    \draw[axis] (tpp cs:0,-0.5,0) -- (tpp cs:0,1.5,0);
                    \draw[axis] (tpp cs:0,0,0) -- (tpp cs:0,0,1.5);

                    \begin{scope}[scale=1.5]
                        \coordinate (E) at (tpp cs:1,0,0.5);
                        \coordinate (F) at (tpp cs:0,0,0.8);
                        \coordinate (G) at (tpp cs:0,1,0.5);
                        \coordinate (H) at (tpp cs:1,1,0.2);
                        \visible<2->{\draw[top color=white, opacity=0.5, bottom color=gray!50] (E) -- (F) -- (G) -- (H)-- cycle;}
                    \end{scope}

                    \filldraw (tpp cs:0.7,0.8,0) circle (1pt) node[below right] {$\mathbf{p}_1$};
                    \visible<1-3>{\filldraw (tpp cs:0.1,0.4,0) circle (1pt) node[above right] {$\mathbf{p}_2$};}
                    \visible<1-2>{\filldraw (tpp cs:0.7,0.3,0.5) circle (1pt) node[above right] {$\mathbf{p}_3$};}
                    \visible<3->{\draw[->, >=Triangle, ultra thick, u] (tpp cs:0.7,0.8,0) -- (tpp cs:0.9,0.3,0.4);}
                    \visible<4->{\draw[->, >=Triangle, ultra thick, v] (tpp cs:0.7,0.8,0) -- (tpp cs:0.1,0.9,0.4);}
                \end{scope}

                \draw[red] (current bounding box.south west) rectangle (current bounding box.north east);
            \end{tikzpicture}
        \end{center}
    \end{columns}
\end{frame}


\begin{frame}{Plane line intersection}

    \begin{columns}

        \column{0.5\textwidth}
        \centering

        \begin{align*}
            \mathbf{p} & = \textcolor{gray!50}{\mathbf{P}} \vee \textcolor{gray}{\mathbf{L}}                \\
                       & = (\textcolor{gray!50}{\mathbf{p}_1 \wedge \mathbf{p}_2 \wedge \mathbf{p}_3}) \vee
            (\textcolor{gray}{\mathbf{p}_4 \wedge \mathbf{p}_5})                                            \\
        \end{align*}



        \column{0.5\textwidth}
        \begin{tikzpicture}[line join=round]
            \useasboundingbox (-3,-3) rectangle (3,2);

            \begin{scope}[
                    scale=1.5,
                    view={55}{135},
                    perspective={
                            p={(-7,0,0)},
                            q={(0,-7,0)},
                            r={(0,0,60)}
                        },
                    bullet/.style={circle,fill,inner sep=1pt},font=\sffamily,
                    cube/.style={very thick,black},
                    grid/.style={opacity=0.1, very thin,gray},
                    vec/.style={->, >=triangle 45, ultra thick, yellow},
                    axis/.style={dotted, thick},
                    bivector/.style={draw=none,fill=yellow, opacity=0.2}]

                % Draw a grid in the x-y plane
                \foreach \x in {-0.5,0,...,1.5}
                \foreach \y in {-0.5,0,...,1.5}
                    {
                        \draw[opacity=0.1, very thin,gray] (tpp cs:\x,-0.5,0) -- (tpp cs:\x,1.5,0);
                        \draw[opacity=0.1, very thin,gray] (tpp cs:-0.5,\y,0) -- (tpp cs:1.5,\y,0);
                    }

                % Draw the axes
                \draw[axis] (tpp cs:-0.5,0,0) -- (tpp cs:1.5,0,0);
                \draw[axis] (tpp cs:0,-0.5,0) -- (tpp cs:0,1.5,0);
                \draw[axis] (tpp cs:0,0,0) -- (tpp cs:0,0,1.5);


                \coordinate (A2) at (tpp cs:2.0,-0.6,-0.5);   % farther behind
                \coordinate (B2) at (tpp cs:-0.6,2.0,1.5);    % farther in front
                \coordinate (I2) at (tpp cs:0.7,0.7,0.5);
                \draw[ultra thick, gray, dotted] (A2) -- (I2);

                \begin{scope}[scale=1.5]
                    \coordinate (E) at (tpp cs:1,0,0.5);
                    \coordinate (F) at (tpp cs:0,0,0.8);
                    \coordinate (G) at (tpp cs:0,1,0.5);
                    \coordinate (H) at (tpp cs:1,1,0.2);

                    % draw line 
                    \draw[top color=white!0, opacity=0.5, bottom color=gray!50] (E) -- (F) -- (G) -- (H)-- cycle;

                \end{scope}

                \draw[ultra thick, gray] (I2) -- (B2);

                \fill (I2) circle (1pt);

            \end{scope}
            \draw[red] (current bounding box.south west) rectangle (current bounding box.north east);

        \end{tikzpicture}
    \end{columns}

\end{frame}